\section{Developing TheHub}

\begin{meta}
    \mitem{Course}{CS-212}
    \mitem{Term}{2011 Fall}
    \mitem{Profs}{Jeff Butera, Chris Perry}
    \mitem{Langs}{Javascript, Python}
    \mitem{Libs}{web.py, jQuery.js, Spine.js}
\end{meta}


\subsection*{Self Eval}
Having previously spent a lot of time learning about python web
frameworks, I spent most of my time learning about the clientside.
Mostly I was using Spine as an ORM around JSON data passed from the
server. Using Spine and jQuery I was able to easily map the data into
existing DOM elements.

In addition, I analyzed and prototyped responsive widget design ideas
(e.g. adding sensible defaults to checkbox filters).


\subsection*{Course Description}
Developing theHub: Students in this course will participate in the
ongoing betahub project: a development and design effort focused on
improving Hampshire's critical online tool, theHub. The term will begin
with an introduction to betahub's Web.py/Javascript system architecture,
code hierarchy, and existing revision control, testing, and release
mechanisms. The class will then transform into a workshop where students
will pursue assignments for theHub that are commensurate with their
backgrounds and abilities. Interested students must have either a
substantial background in computer science (two programming classes
and/or demonstrable Python/Javascript/AJAX experience) or web design (a
minimum of one college-level class and portfolio of HTML/CSS work) or
both, and are strongly encouraged to contact the instructor before the
first class to discuss their candidacy. Prerequisite: Programmers - 2
programming classes; Designers - 1 web design class and a portfolio of
work.
