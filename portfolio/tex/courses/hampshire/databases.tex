\section{Application-Oriented Databases}

\begin{meta}
    \mitem{Course}{CS-154}
    \mitem{Term}{2011 Spring}
    \mitem{Prof}{Jeff Butera}
    \mitem{Langs}{SQL, Javascript}
    \mitem{DBs}{MySQL, PostGres, CouchDB, MongoDB}
\end{meta}

\courseself
This course was not what I expected when I signed up. I was hoping to
get an actual database course like I would have at WPI. The concepts
covered in class were the ones I had known for years from working with
the web.

Instead, I spent a lot of time on directed study. I taught myself more
about views and stored procedures. I ran through some of the "SQL for
Smarties" series and learned a couple of different ways to represent
graphs and trees. At the beginning of the semester I spent a fair amount
of time getting better at map-reduce in CouchDB and MongoDB as a NoSQL
alternative.

Where I failed was the final project. Instead of just creating and
documenting an arbitrary database I tried to challenge myself. I
believe I bounced between trying to build a rules engine in SQL for
twitter users, a trip store with a SPARQL implementation, and a dynamic
hierarchical tag system which could deduce relations. The rules engine
possibly got the farthest but still nowhere near working.

In hindsight, I should have just turned in one of the many databases I
created for my other projects over the semester. I had created at least
two with the lecture recording software, a few to hold arbitrary data as
I created my views and stored procedures. Sadly, I didn't get out of my
own way and demanded too much from myself for a final project.

\loadtex{evals/databases.tex}
