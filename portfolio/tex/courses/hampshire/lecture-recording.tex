\section{Automatic Lecture Recording (2 evals)}

\begin{meta}
    \mitem{Course}{IS-CS-200, IS-CS-200-7}
    \mitem{Term}{2011 Jan, 2011 Spring}
    \mitem{Prof}{Paul Dickson}
    \mitem{Langs}{bash, python, javascript, c++}
    \mitem{Libs}{Django, OpenCV}
\end{meta}

\courseself
The largest portion of my first year at Hampshire was spent working on
Paul's automated lecture recording system. Since I spent my time away
from school learning linux system administration and python, this was a
"final project" in many ways.

Due to filing issues, one of my Spring studies actually took place
during January. With Warshow, I spent Jan term working 10-15 hours a day
to get the system up and running in time for the semester. This involved
writing the capture software, bash/cron jobs, rewriting the web server,
and connecting it all up. It was intense, but in the end we had the
system more or less together by the time courses started.

At first the system had some troubles, but I helped make sure to plan
ahead during our crunch so it was more debugging than reworking. Pretty
soon, it was trivial to add new courses to the system and aside from
checking the status remotely or running a cleanup command, the system
was running smoothly.

That said, the system was quite hacky. It was stuck together with a few
bash scripts, shell based mutex locks that spanned two servers, and
other little nasties scattered throughout. Though working, it wasn't
maintainable. This was fine though, since the system was due for a
reworking that summer.

During the semester, I intended to do more HCI design on the page.
Sadly, the system did not seem far enough along to really analyze how
people were using it or what could make it better.

That isn't to say I wasn't busy. Instead I worked in all the other
facets and did what little research and GUI stuff was available. I did
get to help with the end of semester usability study. Also, since it
was recording classes, the only time we could go in for repairs was at
night, leaving us in a constant "crunch".

While a challenging few months, I learned all about how hard and
precarious it is to work on a production system. I learned how to better
deal with things when they are on fire so it doesn't just catch fire
again next week. I also got much better with the linux OS, specifically
making separate programs play nicely together without a formal IPC. Most
of all, I'm just glad I had already learned to plan for future features
so most of the modifications were simple once things were running.

This project taught me a fair amount about API interfaces: making sure
to document and set everything up with the future in mind so that new
components can just be dropped in with little effort. Early on I added
django commands which proved useful so all it took to create new courses
was a quick ssh command and editing the crontab.

\loadtex{evals/lecture-build}
\loadtex[\vspace{2in}]{evals/lecture-hci}
