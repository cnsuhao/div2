\section{Mobile Computing}

\begin{meta}
    \mitem{Course}{CS-227}
    \mitem{Prof}{Paul Dickson}
    \mitem{Term}{2011 Spring}
    \mitem{Lang}{Java}
    \mitem{Libs}{Android}
\end{meta}

\subsection*{Self Eval}
Another class where my ambition got the best of me. Having used cabana
previously, I knew I wasn't going to use it again if I could help it.
Instead I taught myself how to program Android. Over the course of the
term I got familiar with the software's structure, how to take advantage
of layouts, and more.

However, I never quite settled on a final project. I bounced from a
game, to a webcomic reader, to a Wings Over Amherst ordering store, and
finally a Scrabble Helper. Each project turned out to be too large for
the time I had left in the course so I tried to switch to something
easier, never finding it. In the end, I should have just partially made
one of my ideas and called it a prototype.

Still, I got much better at storyboarding and prototyping for touch
devices. I created a number of reusable and skinnable widgets, as well
as different ways to lay out various menus on the device. Since learning
how to program Android and getting better with touch design was my
primary goal, the course was not a waste.


\subsection*{Course Description}
Mobile Computing: Building Applications for Handheld Devices: Mobile devices
are becoming more prevalent and the demand for applications that can run on
these mobile devices keeps growing. This course will focus on the strengths and
limitations of mobile devices. Students will explore these topics through the
conception and creation of applications for the iPhone/iPod Touch architecture.
By the end of the semester, all students will be developing iPhone applications
and testing them on devices. This course will be hands on and project based.
Prerequisite: Students are required to have at least one semester of
college-level programming in a high-level programming language. This course
satisfied Division I distribution. PRJ
