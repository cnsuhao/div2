\section{Critical Studies of IMGD}

\begin{meta}
    \mitem{Course}{IMGD-1000}
    \mitem{Term}{2006 Fall-A}
    \mitem{Prof}{David Finkel}
\end{meta}

\crscatindex{\IMGD}{Critical Studies}
\crstermindex{2006A}{Critical Studies}

\coursedesc
This course introduces non-technical studies of computer-based
interactive media and games. The course develops a vocabulary for
discussing games and other interactive media, and tools for analyzing
them.

Students are expected to provide written critiques using the critical
approaches presented in the course. The games and other interactive
media critiqued may be commercially available or under development.


\courseself
This course was more like an English course. We played games then
analyzed the plots, emotional response, use of music, exploits, and
more. Unfortunately, the professor did not play games so in the end I
think he got more out of it than any student. This was true for many
IMGD courses since it was the second year of the program.

Despite not learning many new things, the course gave me a chance to
think critically about what I already knew. It was the beginning of
my studies of HCI. I found I was gravitating towards the areas of
game design which directly interacted with users. I concentrated my
studies on difficulty curves, eye catches, and how to illicit emotional
responses.

The most memorable lesson at WPI was from the developers of Titan Quest.
The save button in the game does nothing but display "game saved". They
The inserted it for the users' comfort; save buttons are just expected.
The After hearing that I simply thought, "I want to study \emph{that}."


\subsection*{Grade: A}
