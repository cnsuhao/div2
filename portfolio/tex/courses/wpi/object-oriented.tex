\section{Object-Oriented Design Concepts}

\begin{meta}
    \mitem{Course}{CS-2102}
    \mitem{Term}{2006 Fall-B}
    \mitem{Prof}{George Heineman}
    \mitem{Lang}{Java} 
\end{meta}

\crscatindex{\CS}{Object-Oriented Design Concepts}
\crstermindex{2006B}{Object-Oriented Design Concepts}

\coursedesc
This course introduces students to an object-oriented model of
programming. Building from the design methodology covered in CS 1101/CS
1102, this course shows how programs can be decomposed into classes
and objects. By emphasizing design, this course shows how to implement
small defect-free programs and evaluate design decisions to select an
optimal design under specific assumptions. Topics include inheritance,
exceptions, interface, design by contract, basic design patterns,
and reuse. Students will be expected to design, implement, and debug
object-oriented programs composed of multiple classes and over a variety
of data structures.

\courseself
I had already learned Java in high school for the AP, so most of this
course was a review. Still, I reinforced my understanding of basic data
structures and OOP. Every assignment we did required the creation of
unit tests. By the end of the term, I was much better at TDD.

\coursegrade{A}
