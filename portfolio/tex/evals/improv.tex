\courseeval{HACU-251}{2011F}{The Improvisor's Laboratory}{Other}{improv}


\subsection*{Course Description}
The Improvisor's Laboratory: This is a class for musicians interested in
developing their expressive and creative skills through improvisation.
It is open to all instrumentalists, including voice and electronics.
It is open to students from any musical background. Students will
be challenged to expand their instrumental vocabulary, and to use
these languages in a context of collective improvisation. We will
look at improvisational music making from a multitude of angles,
breaking it down and putting it together again. This is an intensive
course, requiring weekly rehearsals outside of class with small groups,
listening and reading assignments involving periodic papers, and
compositional exercises. Familiarity with traditional musical notation
is required, as we will be exploring the role notated elements play
in an improvisational work. We will be giving a final concert of the
musical pieces develop during the semester. Prerequisite: Musical
Beginnings or Tonal Theory I.

\subsection*{Instructor Narrative}
Alec took a very pro-active role in this ensemble. He made cogent
contributions to the in class rehearsal process. He gave engaged
critiques to his colleagues. Alec has a range of skills as an
instrumentalist, and showed growth during the semester as an improviser
in a soloistic and collective context.

Alec's writing on music showed a comparable level of engagement. He gave
a compelling narrative of the growth of electronic music in current
contexts. I look forward to his ongoing listening and reading in related
areas of music outside his own area of expertise. He is ready to make a
solid expansion of his musical world.
