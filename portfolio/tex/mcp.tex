\chapter{Multiple Cultural Perspective}


\section*{Me and My Bursting Bubble}

Growing up I was only vaguely aware of my inherent social safeties. I
was raised deep in the heart of suburban Connecticut, though not in a
\emph{physically} gated community. Starting in junior high, I attended
Hopkins, a private school founded in 1660 and jokingly called "the Yale
pipeline". Its populace was predictably homogeneous and sheltered, and
while not explicitly intolerant, straying from the norm could chafe the
powers that be. There was never the shadow of a doubt that graduates
would attend a college prestigious enough to reflect well on the school.

% TODO: footnote for RYASAP
% TODO: footnote on Bridgeport.  Post-industrial, recovering from homicide rate 2.2
% TODO: footnote about moonies
Right after graduation I took a gap year to be a full-time Americorps
volunteer in Bridgeport CT, and was placed as a Youth Development
Coordinator for a non-profit called RYASAP. I worked with fellow
volunteers, kids of all ages, high school volunteers from Bridgeport
and the surrounding suburbs, Baptists and "Moonies", after school
programs and youth leadership groups, city and nonprofit employees,
donors and grant recipients, the homeless, those eligible for Title
19, and everyone in between. The people I met not only saw the world
differently, but had backgrounds which varied widely from my own. These
differences were not caused by personality traits or geography; they
arose from racial, socio-economic, and gender marginalization. More than
anything else, I became aware how much of the agency I possess derives
from the color of my skin, where I was born, and how I was raised.

% TODO: footnote for UU
Having been raised a Unitarian Universalist, I was partially prepared
for and open to other world views. But that year, I observed another
sheltered culture where I found people who were not always familiar
with the ideas and concepts I had considered universal. Most devout
people assumed that anyone with morals attended church each Sunday, so I
hid my atheism. Discussions with fellow volunteers about AIDS centered
around conspiracies not facts, so I bit my tongue. Kids never learned
to read LEGO instructions or count tabs, so we played organically. Many
people thought I was Puerto Rican, so I corrected them, saying "I'm
white."

Also during this time, I witnessed things I had only heard of. Listening
to the homeless tell their stories and realizing how transient being
homeless can be. Hearing faith-based educators spout misinformation
about condoms and drugs. Watching a coworker check CT gun laws to see
if he could legally register the gun he just bought. Seeing my fellow
volunteers physically flinch when they discovered someone was gay.
Getting bear hugged after helping out a man just released from prison.
Listening for gunshots or fighting before letting the kids go outside to
play. Witnessing a coworker's pain after the death of her son, stabbed
for teasing another teen. Being speechless when her other son, only
seven, matter-of-factly exclaimed, "he's with Jesus."


\section*{The Information Age Is Coming}

One thing I was not prepared for was how little the internet had
penetrated day to day life in inner city Bridgeport. In high school,
I rarely went more than a day without going online. But the people I
encountered did not. For them, the internet was more of a luxury, a
novelty used by businesses and teens.

Today, the internet is becoming inextricably intertwined with everyday
life, shifting from luxury to necessity. Those without access are
separated from a large portion of our society, inhibiting their social
mobility. Presently, many forces are at work trying to narrow the access
gap. But obtaining access is not enough; the true disparity in power
arises from an individual's inability to navigate the vast ocean of
information to gain knowledge. We need to encourage users to ask questions,
search for answers, and filter information; as a whole, we need to work
towards "internet savviness". Only then can the internet act as a cultural
equalizer, where anyone can be more informed, more connected, and more
heard.


%%% Access Inequity*

Considering its promise, it is unacceptable that a great number of
people have no means of access. Physical or financial, language
or literacy, the entry barriers are numerous. No longer a mere
inconvenience, lack of access leads to disenfranchisement. As more
people connect through social networks, those who cannot lose their
voice. When job listings are online, those without access have more
trouble finding work. The internet is already a necessity for students.
Online only services may leave out those who could benefit most.

Most people connect through service providers whose rates are calculated
as if access is a luxury. There is a need for lower or subsidized rates
so more people can afford to go online at home without being unduly
burdened. This is why a great deal of effort and resources are now being
focused on making the internet nationally available. Public grants
and corporate investments fund computers in schools and libraries,
broadband build-outs in rural areas, and free WiFi in cities. The costs
of consumer electronics and high-speed access have dropped, and most
cellphones are internet capable. In addition the number and variety
of services have grown to accommodate the rising demand. Consequently,
internet penetration and usage has skyrocketed over the last few years.


%%% Savviness*

Unfortunately, everyone on the internet is a target for social engineers
looking to prey on others. Though anyone can fall victim to phishing,
scams, "too good to be true" deals, and identity theft, the impact can
be more devastating on those less fortunate. However, savvy users can
recognize when something "feels wrong", making them less likely to be
victimized. Also, they know how to use multiple sources to distinguish
fact from fiction, letting them see through misinformation. They
are able to see a larger picture drawn from filtering irrelevant or
incorrect information. They are more proficient at selecting search
terms, yielding more accurate results. They know how to navigate the
web: where they are, how they got there, and how to return. They know
how to find answers to their own questions, taking advantage of others'
expertise. Their "savviness" comes from knowing how to fish, not just
asking for one.


%%% Exploitation*

Many services on the internet are free; anyone can sign up and
immediately use email, Facebook, Twitter, and innumerable other online
communities. These connections have grown into a standard form of
communication and are becoming de rigueur, vital for keeping in touch
with friends, family, and the rest of the world. This growing need to
be part of the 24-7 online discourse is driving smartphone sales and
usage. Unfortunately, these devices and the data plans they require
are expensive, costing at least three times as much as a landline.
Even the most affordable data plans are capped, leading to significant
overage charges for unsuspecting users. 

Owners may subscribe to a variety of additional services, for yet
another monthly fee. Individual songs, videos, books, and apps are
priced to seem cheap but quickly add up, surprising many when the bill
comes. Apps advertised as free can require an additional purchase to
unlock the full functionality. Pay-to-play games offer shortcuts in
exchange for every nickel and dime. At the end of the month, additional
luxury purchases have eclipsed the anticipated financial burden; what
started as a necessity quickly becomes unaffordable.


%%% Conclusion*

The exploding internet population will continue to face challenges
surrounding access, inexperience, and exploitation. Social policies
will attempt to reduce access inequity and exploitation. But it is up
to individuals to begin a cultural shift to "internet savviness". Be
patient when people are uninformed, taking the time to share personal
experience. Encourage new users to examine less expensive solutions
that still meet their need. Advocate alternative software and services,
especially ones which are free and/or open source. Encourage people
to question what they are told, to protect themselves from becoming
victims. Show people how to find answers instead of just spoon feeding
them. As individuals, we should share our knowledge to help others
become more savvy and reduce the power disparity between users.

Personally, I have tried to hold to these ideals in my own life: as a
coder, IT technician, nonprofit worker, and now aspiring educator.
% personal examples
