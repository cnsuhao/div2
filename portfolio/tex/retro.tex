\chapter{Retrospective}

\section*{Concentration}

% TODO: minimally, explain what I did...done
% TODO: venue I should use to say what should be said
% TODO: reflect on what it meant to ___

Human-Computer Interaction (HCI) has always been where my passion lies.
While I started out pursuing game development, my interests rapidly
evolved into interactive media and interfaces in general. By interface I
do not just mean GUIs; more often I design programmer interfaces: APIs,
DSLs, scripts, and frameworks.

For the last few years I concentrated on finding ways to close the
semantic gap between how humans and computers "think". I attempt
to create more intuitive, automated, and modular interfaces. I develop
DSLs and APIs to write software in the problem's domain instead of the
language or system's logic. In addition, I have been looking at the
problem from the other side; teaching others to program.


\section*{Areas of Study}
\subsection*{Programming}

Every so often I hear Computer Science majors/concentrators exclaim that
they are great at Computer Science but bad at programming. I am left
baffled, as if a painter told me they cannot mix colors. Although my
interests can be academic and I am considering becoming a professor, at
the end of the day I am a programmer first. I take pride in being a code
monkey and put a lot of effort into honing my craft.

To that end, I primarily studied the art of programming, taking a
liberal arts approach to my study of Computer Science. In addition
to core CS courses, I studied a broad range of topics, many simply
out of curiosity. I learned a variety of programming languages\footnote{e.g.
Python, Clojure, Java, C/C++} and through them gained a strong
understanding of how to work within various paradigms.\footnote{e.g. functional,
object-oriented, procedural, declarative} The majority of my academic work has also
involved working with project teams, teaching me to communicate and
organize my ideas better.

In addition to my academic work, I spend most of my free time on
personal study and programming projects. Since I started programming in
high school, I doubt I have gone more than a week without programming
something. I try to learn a little bit about everything I come across;
the areas I study can be ecclectic at times.\footnote{e.g. formal grammars,
lambda calculus, tripstores} I spend a lot of time mastering my
tools\footnote{e.g. IDEs, Linux, debuggers, version control, issue trackers} and
learning pragmatic techniques.\footnote{e.g. debugging, logging, unit tests, prefactoring,
iterative development, spec gathering} I also
regularly create new tools to improve my development environment or
automate my life.\footnote{e.g. shell scripts, cronjobs, vim plugins} At this
point, I am far more likely to curl up on a couch with source code than
a novel.

This course of study has given me a strong foundation in software
development. The largest and most challenging software project I tackled
at Hampshire was the lecture recording system with Paul Dickson. During
my time on the team I developed a web frontend, setup the backend,
and continuously debugged the live system. All the while we continued
developing the next iteration of the software. This project drew upon
my skills in various languages, interface design, Linux, and sleep
deprivation. The work paid off, and the system recorded a full semester's
worth of courses for the first time.

At this point, I am on the path to being a "guru"; I have started to
develop an intuition and fluency in designing software.


\subsection*{Interfaces}
My passion in HCI stemmed from of my passion for video games. More
than other interfaces, games need to be incredibly instructive and
responsive. I entered WPI as a Technical IMGD major (Interactive
Media and Game Design) and studied a wide range of game design topics
including puzzle design, player experience and motivation, multiplayer
interaction, interactive storytelling, and game balance. As a tech major
I also studied game engine architecture, AI, and graphics, ultimately
creating games and artist tools. Additionally, I was an active member of
our Game Development Club (GDC), and was elected President in my last
year at WPI.

Eventually, my interest in game development evolved into a more general
interest in how users interact with computers. I took my skills in game
creation and started applying them to creating more interactive web and
mobile applications. During my time between schools, I taught myself
various web backends.\footnote{e.g. django, pylons, PHP, GAE, CGI} I tried
to keep up with the changing HTML5 and CSS3 specs as well as a wide
assortment of javascript libraries and practices. In addition, I taught
myself as much as I could about developing Android applications and
their structure.

Over the years I learned HCI fundamentals such as interface design
analysis, storyboards, affordances, and usability studies. In my
projects I try to factor out any reusable libraries, plugins, and
widgets as I go. More recently I have been studying technical user
interface design including frameworks, APIs, and DSLs.


\subsection*{Instruction}

Before I went to WPI, I took a year off to volunteer for Americorps in
Bridgeport, CT. I worked at RYASAP as a Youth Development Coordinator
working with youth from inner city Bridgeport and the surrounding
suburbs. In my time there, I had my fingers in many pies; I tutored
grade schoolers in a public housing after school program, helped plan
and run a youth leadership development week for high schoolers, was
involved with youth groups of all ages, sat in on various boards, wrote
and approved grassroot grants, worked with homeless shelters, and helped
out around the office.

Even years later I can only describe the experience as undescribable.
I was thrown into situations with individuals who perceived the
world radically differently than I. These days, I try to make fewer
assumptions and promote the voices of those involved rather than simply
raising my own. More than anything else I’ve come to appreciate
listening more than talking. 

When it comes to technology, I have always
found myself in the role of instructor. Whether it is people asking
for help with computers or me rambling excitedly about CS concepts, I
love sharing my knowledge with others. At WPI I more formally pursued
IT support as a helpdesk employee, serving as a mouthpiece for the
larger IT department. Between schools I continued this path working as
a support technician in an ISP call center. Over time these jobs have
helped me radically improve my patience and persistence. They have given
me the ability to explain myself and understand the perspectives and
confusions of nontechnical people.

While I enjoy helping people with their computer problems, I am driven
to teach people to program. Originally, I would just help my friends
and peers every chance I could with their programming projects. A
large part of my involvement in WPI’s GDC was running crash courses
in pygame, teaching freshmen game design concepts, and supporting the
development teams however possible. Starting last year, I took every
TA opportunity I could and tried to make myself available for anyone
with a programming question on campus. This culminated in what I (half)
jokingly call my first Division 3. Taking everything I learned as a CS
student and TA, I developed a CS1 course in Python and game development
for non-programmers with Paul Dickson (2012S CS-114). I created the
overall curriculum, the in-class examples, sample code, homework
assignments, homework submission and grading tools, and taught/assisted
in every class. In addition, I taught the material to the other TAs and
professor, held lab hours 3 times a week (often lasting 5 hours each),
and prepared/taught a section on Fridays for the advanced students.
In the end, I have a whole new appreciation and understanding on how
to develop and run a course. This has only increased my appetite for
teaching.

\subsection*{Outcome}

\subsubsection{Division III}

As a final project in my studies of Human Computer Interaction I
intend to design and develop a touch-based, visual scripting environment
(TVSE). As a practical case, it will allow users to program a robotics
kit on the Android platform.

\subsubsection*{Conclusion}
It has been over a decade since I wrote my first few lines of code and
about 4-5 years since I completed my ”Division I” at WPI. Despite
the sometimes rocky path I took, I am proud of what I have learned and
accomplished in that time. I have studied how to create a wide range
of interfaces for a wide range of media. I improved on my fundamental
skills as a coder. I consumed as much information on as many CS topics
as I could. Not content to remain a simple code monkey, I have been
trying to improve as an instructor.

Most importantly, to me anyway, I feel I have done and learned what
I can as a Division II student, and look forward to what I will be
learning next.
