\section{Technical Game Development 1}

\begin{meta}
    \mitem{Course}{IMGD-3000}
    \mitem{Term}{2007 Fall-B}
    \mitem{Prof}{Rob Lindeman}
    \mitem{Lang}{C++}
    \mitem{Lib}{C4 (3D Game Engine)}
    \mitem{Tools}{Visual Studio 8}
\end{meta}

\coursedesc
This course teaches technical Computer Science aspects of game
development, with the focus of the course on low-level programming of a
computer games. Topics include 2D and 3D game engines, simulation-type
games, analog and digital controllers and other forms of tertiary input.
Students will implement games or parts of games, including exploration
of graphics, sound, and music as it affects game implementation.


\courseself
In this course we worked in teams of four: two from Tech Game Dev 1, two
from Artistic Game Dev 1. We were programming Windows games using the
C4 engine. For our project, my group created a 2-1/2d sidescroller. Our
project was an overall success, winning "best tech" for the class.

The largest piece of my work was a camera system. The games camera could
easily be set from triggers in the level editor. In addition, there
were trigger regions which splined the camera according to the player's
position. This made it easy for the artists to get the exact camera
angle they wanted for each section of a level.

This project was the first time I noticed how much effort I put into
designing interfaces versus game design itself. Creating useful tools
for the artists was much more rewarding than hard coding something that
worked. It was the final revelation that made me leave game design for
the more general concentration of HCI.


\coursegrade{A}
